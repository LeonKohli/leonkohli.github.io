%% Generated by Sphinx.
\def\sphinxdocclass{report}
\documentclass[letterpaper,10pt,ngerman]{sphinxmanual}
\ifdefined\pdfpxdimen
   \let\sphinxpxdimen\pdfpxdimen\else\newdimen\sphinxpxdimen
\fi \sphinxpxdimen=.75bp\relax
\ifdefined\pdfimageresolution
    \pdfimageresolution= \numexpr \dimexpr1in\relax/\sphinxpxdimen\relax
\fi
%% let collapsible pdf bookmarks panel have high depth per default
\PassOptionsToPackage{bookmarksdepth=5}{hyperref}

\PassOptionsToPackage{warn}{textcomp}
\usepackage[utf8]{inputenc}
\ifdefined\DeclareUnicodeCharacter
% support both utf8 and utf8x syntaxes
  \ifdefined\DeclareUnicodeCharacterAsOptional
    \def\sphinxDUC#1{\DeclareUnicodeCharacter{"#1}}
  \else
    \let\sphinxDUC\DeclareUnicodeCharacter
  \fi
  \sphinxDUC{00A0}{\nobreakspace}
  \sphinxDUC{2500}{\sphinxunichar{2500}}
  \sphinxDUC{2502}{\sphinxunichar{2502}}
  \sphinxDUC{2514}{\sphinxunichar{2514}}
  \sphinxDUC{251C}{\sphinxunichar{251C}}
  \sphinxDUC{2572}{\textbackslash}
\fi
\usepackage{cmap}
\usepackage[T1]{fontenc}
\usepackage{amsmath,amssymb,amstext}
\usepackage{babel}



\usepackage{tgtermes}
\usepackage{tgheros}
\renewcommand{\ttdefault}{txtt}



\usepackage[Sonny]{fncychap}
\ChNameVar{\Large\normalfont\sffamily}
\ChTitleVar{\Large\normalfont\sffamily}
\usepackage{sphinx}

\fvset{fontsize=auto}
\usepackage{geometry}


% Include hyperref last.
\usepackage{hyperref}
% Fix anchor placement for figures with captions.
\usepackage{hypcap}% it must be loaded after hyperref.
% Set up styles of URL: it should be placed after hyperref.
\urlstyle{same}

\addto\captionsngerman{\renewcommand{\contentsname}{Contents:}}

\usepackage{sphinxmessages}
\setcounter{tocdepth}{1}



\title{MeineDoku}
\date{28.09.2022}
\release{1}
\author{Leon}
\newcommand{\sphinxlogo}{\vbox{}}
\renewcommand{\releasename}{Release}
\makeindex
\begin{document}

\ifdefined\shorthandoff
  \ifnum\catcode`\=\string=\active\shorthandoff{=}\fi
  \ifnum\catcode`\"=\active\shorthandoff{"}\fi
\fi

\pagestyle{empty}
\sphinxmaketitle
\pagestyle{plain}
\sphinxtableofcontents
\pagestyle{normal}
\phantomsection\label{\detokenize{index::doc}}


\begin{sphinxShadowBox}
\sphinxstylesidebartitle{Hausboot}

\noindent\sphinxincludegraphics{{hausboot}.jpg}
\end{sphinxShadowBox}

\sphinxstepscope


\chapter{Publizieren für ZEAL}
\label{\detokenize{ueberblick:publizieren-fur-zeal}}\label{\detokenize{ueberblick::doc}}
\sphinxAtStartPar
Was alles benötigt wird, damit meine Kollegen
nur Zeal und die Docset installieren muss

\noindent\sphinxincludegraphics{{workflow}.png}

\sphinxstepscope


\chapter{Emacs wichtige Befehle!}
\label{\detokenize{emacs/emacs-basics:emacs-wichtige-befehle}}\label{\detokenize{emacs/emacs-basics::doc}}

\section{Tastenkombinationen}
\label{\detokenize{emacs/emacs-basics:tastenkombinationen}}

\begin{savenotes}\sphinxattablestart
\centering
\begin{tabular}[t]{|*{2}{\X{1}{2}|}}
\hline
\sphinxstyletheadfamily 
\sphinxAtStartPar
Tastenkombination
&\sphinxstyletheadfamily 
\sphinxAtStartPar
Anmerkung
\\
\hline
\sphinxAtStartPar
\sphinxkeyboard{\sphinxupquote{Ctrl\sphinxhyphen{}x Ctrl\sphinxhyphen{}s}}
&
\sphinxAtStartPar
Änderungen speichern
\\
\hline
\sphinxAtStartPar
\sphinxkeyboard{\sphinxupquote{Ctrl\sphinxhyphen{}g}}
&
\sphinxAtStartPar
3x == Abbruch
\\
\hline
\sphinxAtStartPar
\sphinxkeyboard{\sphinxupquote{Ctrl\sphinxhyphen{}k}}
&
\sphinxAtStartPar
Löscht Zeile
\\
\hline
\sphinxAtStartPar
\sphinxkeyboard{\sphinxupquote{Ctrl\sphinxhyphen{}a}}
&
\sphinxAtStartPar
Anfang Zeile
\\
\hline
\sphinxAtStartPar
\sphinxkeyboard{\sphinxupquote{Ctrl\sphinxhyphen{}e}}
&
\sphinxAtStartPar
Ende der Zeile
\\
\hline
\sphinxAtStartPar
\sphinxkeyboard{\sphinxupquote{Ctrl\sphinxhyphen{}k}}
&\begin{enumerate}
\sphinxsetlistlabels{\arabic}{enumi}{enumii}{}{.}%
\item {} 
\sphinxAtStartPar
Zeile Löschen

\end{enumerate}
\\
\hline
\sphinxAtStartPar
\sphinxkeyboard{\sphinxupquote{Ctrl\sphinxhyphen{}y}}
&\begin{enumerate}
\sphinxsetlistlabels{\arabic}{enumi}{enumii}{}{.}%
\setcounter{enumi}{1}
\item {} 
\sphinxAtStartPar
beliebig oft einfügen

\end{enumerate}
\\
\hline
\sphinxAtStartPar
\sphinxkeyboard{\sphinxupquote{Ctrl\sphinxhyphen{}x Ctrl\sphinxhyphen{}f}}
&
\sphinxAtStartPar
Find File \sphinxhyphen{} Neue Datei
\\
\hline
\sphinxAtStartPar
\sphinxkeyboard{\sphinxupquote{Ctrl\sphinxhyphen{}x Ctrl\sphinxhyphen{}1}}
&
\sphinxAtStartPar
aktuelles Fesnter im Vollbild
\\
\hline
\sphinxAtStartPar
\sphinxkeyboard{\sphinxupquote{Ctrl\sphinxhyphen{}x Ctrl\sphinxhyphen{}2}}
&
\sphinxAtStartPar
splittet horizontal
\\
\hline
\sphinxAtStartPar
\sphinxkeyboard{\sphinxupquote{Ctrl\sphinxhyphen{}x Ctrl\sphinxhyphen{}3}}
&
\sphinxAtStartPar
splittet vertikal
\\
\hline
\sphinxAtStartPar
\sphinxkeyboard{\sphinxupquote{Ctrl\sphinxhyphen{}w}}
&
\sphinxAtStartPar
markierten Text Löschen
\\
\hline
\sphinxAtStartPar
\sphinxkeyboard{\sphinxupquote{Ctrl\sphinxhyphen{}Space}}
&
\sphinxAtStartPar
Start markieren
\\
\hline
\end{tabular}
\par
\sphinxattableend\end{savenotes}


\section{Linksammlung}
\label{\detokenize{emacs/emacs-basics:linksammlung}}
\sphinxAtStartPar
Scratchpad: \sphinxurl{https://livesphinx.herokuapp.com/}

\sphinxAtStartPar
Emacs Download: \sphinxurl{https://www.gnu.org/software/emacs/}

\sphinxAtStartPar
Cheat\sphinxhyphen{}Sheets: \sphinxurl{http://www.cheat-sheets.org/}

\sphinxAtStartPar
Suchmaschiene: \sphinxurl{https://metager.de/}

\sphinxAtStartPar
GitHub: \sphinxurl{https://GitHub.com}

\sphinxstepscope


\chapter{Software}
\label{\detokenize{software/index:software}}\label{\detokenize{software/index::doc}}
\sphinxstepscope


\section{Versionsverwaltung mit GIT}
\label{\detokenize{software/git:versionsverwaltung-mit-git}}\label{\detokenize{software/git::doc}}

\subsection{Neues Repository anlegen}
\label{\detokenize{software/git:neues-repository-anlegen}}
\sphinxAtStartPar
Einmalig für den Projektstart

\begin{sphinxVerbatim}[commandchars=\\\{\}]
\PYG{n}{cd} \PYG{o}{\PYGZlt{}}\PYG{n}{mein}\PYG{o}{/}\PYG{n}{zielordner}\PYG{o}{\PYGZgt{}}
\PYG{n}{git} \PYG{n}{init} \PYG{o}{\PYGZhy{}}\PYG{o}{\PYGZhy{}}\PYG{n}{bare}
\PYG{c+c1}{\PYGZsh{} Initialized empty Git repository in}
\PYG{c+c1}{\PYGZsh{} //BEFS\PYGZhy{}00188\PYGZhy{}043.sz.zitbb.lvnbb.de/sz\PYGZhy{}daten\PYGZdl{}/Raum8/repos/name/}
\end{sphinxVerbatim}


\subsection{Kopie eines Repos}
\label{\detokenize{software/git:kopie-eines-repos}}
\sphinxAtStartPar
Einmalig auf lokalem Laufwerk

\begin{sphinxVerbatim}[commandchars=\\\{\}]
\PYG{n}{git} \PYG{n}{clone} \PYG{n}{O}\PYG{p}{:}\PYGZbs{}\PYG{n}{Raum8}\PYGZbs{}\PYG{n}{repos}\PYGZbs{}\PYG{n}{janneck}
\PYG{n}{git} \PYG{n}{config} \PYG{o}{\PYGZhy{}}\PYG{o}{\PYGZhy{}}\PYG{k}{global} \PYG{n}{user}\PYG{o}{.}\PYG{n}{name} \PYG{n}{Janneck}
\PYG{n}{git} \PYG{n}{config} \PYG{o}{\PYGZhy{}}\PYG{o}{\PYGZhy{}}\PYG{k}{global} \PYG{n}{user}\PYG{o}{.}\PYG{n}{email} \PYG{n}{janneck}\PYG{o}{.}\PYG{n}{lehmann}\PYG{n+nd}{@zit}\PYG{o}{\PYGZhy{}}\PYG{n}{bb}\PYG{o}{.}\PYG{n}{brandenburg}\PYG{o}{.}\PYG{n}{de}
\end{sphinxVerbatim}


\subsection{Jeden Tag, wenn neue Inhalte entstehen}
\label{\detokenize{software/git:jeden-tag-wenn-neue-inhalte-entstehen}}
\begin{sphinxVerbatim}[commandchars=\\\{\}]
\PYG{n}{git} \PYG{n}{status} \PYG{c+c1}{\PYGZsh{} was hat sich geändert}
\PYG{n}{git} \PYG{n}{add} \PYG{o}{.}  \PYG{c+c1}{\PYGZsh{} alles neue hinzufügen zum \PYGZdq{}Stash\PYGZdq{}}
\PYG{n}{git} \PYG{n}{reset}  \PYG{c+c1}{\PYGZsh{} wenn die Liste nicht korrekt ist}
           \PYG{c+c1}{\PYGZsh{} meist unerwünschte Dateien und Ordner}
\PYG{n}{git} \PYG{n}{commit} \PYG{o}{\PYGZhy{}}\PYG{n}{m} \PYG{l+s+s2}{\PYGZdq{}}\PYG{l+s+s2}{ein einzeilieger Kommentar}\PYG{l+s+s2}{\PYGZdq{}}
\PYG{n}{git} \PYG{n}{pull}
\PYG{n}{git} \PYG{n}{push}
\end{sphinxVerbatim}


\subsection{Clone im ZIT nur mit Proxy}
\label{\detokenize{software/git:clone-im-zit-nur-mit-proxy}}
\index{Proxy@\spxentry{Proxy}}\ignorespaces 
\def\sphinxLiteralBlockLabel{\label{\detokenize{software/git:index-0}}}
\begin{sphinxVerbatim}[commandchars=\\\{\}]
\PYG{n+nb}{set} \PYG{n}{http\PYGZus{}proxy}\PYG{o}{=}\PYG{l+m+mf}{10.128}\PYG{l+m+mf}{.9}\PYG{l+m+mf}{.30}\PYG{p}{:}\PYG{l+m+mi}{80}
\PYG{n+nb}{set} \PYG{n}{https\PYGZus{}proxy}\PYG{o}{=}\PYG{l+m+mf}{10.128}\PYG{l+m+mf}{.9}\PYG{l+m+mf}{.30}\PYG{p}{:}\PYG{l+m+mi}{80}
\end{sphinxVerbatim}

\sphinxstepscope


\section{Sphinx}
\label{\detokenize{software/sphinx:sphinx}}\label{\detokenize{software/sphinx::doc}}

\subsection{Installation}
\label{\detokenize{software/sphinx:installation}}
\begin{DUlineblock}{0em}
\item[] \sphinxhyphen{} Phyton instalieren (3.10)
\item[]
\begin{DUlineblock}{\DUlineblockindent}
\item[] \sphinxurl{https://www.python.org/downloads/}
\end{DUlineblock}
\item[] \sphinxhyphen{} Anpassungen:
\item[]
\begin{DUlineblock}{\DUlineblockindent}
\item[] \sphinxhyphen{} Pfadkürzung auf: C:/Python310
\item[] \sphinxhyphen{} Tick bei „…enviroment variables“
\end{DUlineblock}
\end{DUlineblock}


\subsection{Virtuelle Enviroments}
\label{\detokenize{software/sphinx:virtuelle-enviroments}}

\subsubsection{Einmalig Einrichten}
\label{\detokenize{software/sphinx:einmalig-einrichten}}
\begin{sphinxVerbatim}[commandchars=\\\{\}]
\PYG{n}{cd} \PYG{n}{C}\PYG{p}{:}\PYGZbs{}\PYG{n}{home}\PYGZbs{}\PYG{n}{dokus}\PYGZbs{}\PYG{n}{peter}
\PYG{n}{py} \PYG{o}{\PYGZhy{}}\PYG{n}{m} \PYG{n}{venv} \PYG{n}{env}
\end{sphinxVerbatim}

\sphinxAtStartPar
Aktivieren: Immer vor Arbeitsstart!!!!
Beachte geänderte Prompt!!!! (env)


\subsubsection{Aktivieren}
\label{\detokenize{software/sphinx:aktivieren}}
\begin{sphinxVerbatim}[commandchars=\\\{\}]
\PYG{n}{cd} \PYG{n}{C}\PYG{p}{:}\PYGZbs{}\PYG{n}{home}\PYGZbs{}\PYG{n}{dokus}\PYGZbs{}\PYG{n}{leon}
\PYG{o}{.}\PYGZbs{}\PYG{n}{env}\PYGZbs{}\PYG{n}{Scripts}\PYGZbs{}\PYG{n}{activate}\PYG{o}{.}\PYG{n}{bat}
\end{sphinxVerbatim}


\subsubsection{Dokumentation mit Sphinx}
\label{\detokenize{software/sphinx:dokumentation-mit-sphinx}}
\begin{sphinxVerbatim}[commandchars=\\\{\}]
\PYG{n}{pip} \PYG{n}{install} \PYG{n}{sphinx}
\PYG{n}{pip} \PYG{n}{install} \PYG{n}{sphinx}\PYG{o}{\PYGZhy{}}\PYG{n}{autobuild}
\PYG{n}{pip} \PYG{n}{install} \PYG{n}{furo}
\end{sphinxVerbatim}


\paragraph{manuelle Version}
\label{\detokenize{software/sphinx:manuelle-version}}\begin{quote}

\begin{sphinxVerbatim}[commandchars=\\\{\}]
\PYG{n}{make} \PYG{n}{html}
\end{sphinxVerbatim}
\end{quote}


\paragraph{automatische Übersetzung}
\label{\detokenize{software/sphinx:automatische-ubersetzung}}\begin{quote}

\begin{sphinxVerbatim}[commandchars=\\\{\}]
\PYG{o}{.}\PYGZbs{}\PYG{n}{env}\PYGZbs{}\PYG{n}{Scripts}\PYGZbs{}\PYG{n}{sphinx}\PYG{o}{\PYGZhy{}}\PYG{n}{autobuild} \PYG{n}{source} \PYG{n}{build}
\end{sphinxVerbatim}
\end{quote}


\subsection{Konfiguration}
\label{\detokenize{software/sphinx:konfiguration}}
\begin{sphinxVerbatim}[commandchars=\\\{\}]
\PYG{c+c1}{\PYGZsh{} Configuration file for the Sphinx documentation builder.}
\PYG{c+c1}{\PYGZsh{}}
\PYG{c+c1}{\PYGZsh{} For the full list of built\PYGZhy{}in configuration values, see the documentation:}
\PYG{c+c1}{\PYGZsh{} https://www.sphinx\PYGZhy{}doc.org/en/master/usage/configuration.html}

\PYG{c+c1}{\PYGZsh{} \PYGZhy{}\PYGZhy{} Project information \PYGZhy{}\PYGZhy{}\PYGZhy{}\PYGZhy{}\PYGZhy{}\PYGZhy{}\PYGZhy{}\PYGZhy{}\PYGZhy{}\PYGZhy{}\PYGZhy{}\PYGZhy{}\PYGZhy{}\PYGZhy{}\PYGZhy{}\PYGZhy{}\PYGZhy{}\PYGZhy{}\PYGZhy{}\PYGZhy{}\PYGZhy{}\PYGZhy{}\PYGZhy{}\PYGZhy{}\PYGZhy{}\PYGZhy{}\PYGZhy{}\PYGZhy{}\PYGZhy{}\PYGZhy{}\PYGZhy{}\PYGZhy{}\PYGZhy{}\PYGZhy{}\PYGZhy{}\PYGZhy{}\PYGZhy{}\PYGZhy{}\PYGZhy{}\PYGZhy{}\PYGZhy{}\PYGZhy{}\PYGZhy{}\PYGZhy{}\PYGZhy{}\PYGZhy{}\PYGZhy{}\PYGZhy{}\PYGZhy{}\PYGZhy{}\PYGZhy{}\PYGZhy{}\PYGZhy{}}
\PYG{c+c1}{\PYGZsh{} https://www.sphinx\PYGZhy{}doc.org/en/master/usage/configuration.html\PYGZsh{}project\PYGZhy{}information}

\PYG{n}{project} \PYG{o}{=} \PYG{l+s+s1}{\PYGZsq{}}\PYG{l+s+s1}{MeineDoku}\PYG{l+s+s1}{\PYGZsq{}}
\PYG{n}{copyright} \PYG{o}{=} \PYG{l+s+s1}{\PYGZsq{}}\PYG{l+s+s1}{2022, Leon}\PYG{l+s+s1}{\PYGZsq{}}
\PYG{n}{html\PYGZus{}title} \PYG{o}{=} \PYG{l+s+s1}{\PYGZsq{}}\PYG{l+s+s1}{\PYGZsq{}}
\PYG{n}{author} \PYG{o}{=} \PYG{l+s+s1}{\PYGZsq{}}\PYG{l+s+s1}{Leon}\PYG{l+s+s1}{\PYGZsq{}}
\PYG{n}{release} \PYG{o}{=} \PYG{l+s+s1}{\PYGZsq{}}\PYG{l+s+s1}{1}\PYG{l+s+s1}{\PYGZsq{}}
\PYG{n}{html\PYGZus{}logo} \PYG{o}{=} \PYG{l+s+s1}{\PYGZsq{}}\PYG{l+s+s1}{\PYGZus{}static/meinlogo.png}\PYG{l+s+s1}{\PYGZsq{}}
\PYG{n}{html\PYGZus{}favicon} \PYG{o}{=} \PYG{l+s+s1}{\PYGZsq{}}\PYG{l+s+s1}{\PYGZus{}static/meinlogo.png}\PYG{l+s+s1}{\PYGZsq{}}

\PYG{c+c1}{\PYGZsh{} \PYGZhy{}\PYGZhy{} General configuration \PYGZhy{}\PYGZhy{}\PYGZhy{}\PYGZhy{}\PYGZhy{}\PYGZhy{}\PYGZhy{}\PYGZhy{}\PYGZhy{}\PYGZhy{}\PYGZhy{}\PYGZhy{}\PYGZhy{}\PYGZhy{}\PYGZhy{}\PYGZhy{}\PYGZhy{}\PYGZhy{}\PYGZhy{}\PYGZhy{}\PYGZhy{}\PYGZhy{}\PYGZhy{}\PYGZhy{}\PYGZhy{}\PYGZhy{}\PYGZhy{}\PYGZhy{}\PYGZhy{}\PYGZhy{}\PYGZhy{}\PYGZhy{}\PYGZhy{}\PYGZhy{}\PYGZhy{}\PYGZhy{}\PYGZhy{}\PYGZhy{}\PYGZhy{}\PYGZhy{}\PYGZhy{}\PYGZhy{}\PYGZhy{}\PYGZhy{}\PYGZhy{}\PYGZhy{}\PYGZhy{}\PYGZhy{}\PYGZhy{}\PYGZhy{}\PYGZhy{}}
\PYG{c+c1}{\PYGZsh{} https://www.sphinx\PYGZhy{}doc.org/en/master/usage/configuration.html\PYGZsh{}general\PYGZhy{}configuration}

\PYG{n}{extensions} \PYG{o}{=} \PYG{p}{[}\PYG{p}{]}
\PYG{n}{templates\PYGZus{}path} \PYG{o}{=} \PYG{p}{[}\PYG{l+s+s1}{\PYGZsq{}}\PYG{l+s+s1}{\PYGZus{}templates}\PYG{l+s+s1}{\PYGZsq{}}\PYG{p}{]}
\PYG{n}{exclude\PYGZus{}patterns} \PYG{o}{=} \PYG{p}{[}\PYG{p}{]}

\PYG{n}{language} \PYG{o}{=} \PYG{l+s+s1}{\PYGZsq{}}\PYG{l+s+s1}{de}\PYG{l+s+s1}{\PYGZsq{}}

\PYG{c+c1}{\PYGZsh{} \PYGZhy{}\PYGZhy{} Options for HTML output \PYGZhy{}\PYGZhy{}\PYGZhy{}\PYGZhy{}\PYGZhy{}\PYGZhy{}\PYGZhy{}\PYGZhy{}\PYGZhy{}\PYGZhy{}\PYGZhy{}\PYGZhy{}\PYGZhy{}\PYGZhy{}\PYGZhy{}\PYGZhy{}\PYGZhy{}\PYGZhy{}\PYGZhy{}\PYGZhy{}\PYGZhy{}\PYGZhy{}\PYGZhy{}\PYGZhy{}\PYGZhy{}\PYGZhy{}\PYGZhy{}\PYGZhy{}\PYGZhy{}\PYGZhy{}\PYGZhy{}\PYGZhy{}\PYGZhy{}\PYGZhy{}\PYGZhy{}\PYGZhy{}\PYGZhy{}\PYGZhy{}\PYGZhy{}\PYGZhy{}\PYGZhy{}\PYGZhy{}\PYGZhy{}\PYGZhy{}\PYGZhy{}\PYGZhy{}\PYGZhy{}\PYGZhy{}\PYGZhy{}}
\PYG{c+c1}{\PYGZsh{} https://www.sphinx\PYGZhy{}doc.org/en/master/usage/configuration.html\PYGZsh{}options\PYGZhy{}for\PYGZhy{}html\PYGZhy{}output}

\PYG{n}{html\PYGZus{}theme} \PYG{o}{=} \PYG{l+s+s1}{\PYGZsq{}}\PYG{l+s+s1}{furo}\PYG{l+s+s1}{\PYGZsq{}}
\PYG{c+c1}{\PYGZsh{}html\PYGZus{}theme = \PYGZsq{}alabaster\PYGZsq{}}


\PYG{n}{html\PYGZus{}static\PYGZus{}path} \PYG{o}{=} \PYG{p}{[}\PYG{l+s+s1}{\PYGZsq{}}\PYG{l+s+s1}{\PYGZus{}static}\PYG{l+s+s1}{\PYGZsq{}}\PYG{p}{]}
\end{sphinxVerbatim}

\sphinxstepscope


\section{meine Lieblingsvideos}
\label{\detokenize{software/sphinxvideo:meine-lieblingsvideos}}\label{\detokenize{software/sphinxvideo::doc}}


\sphinxstepscope


\section{VS\sphinxhyphen{}Codium}
\label{\detokenize{software/vscodium:vs-codium}}\label{\detokenize{software/vscodium::doc}}
\index{VS\sphinxhyphen{}Codium@\spxentry{VS\sphinxhyphen{}Codium}}\ignorespaces 
\index{Installation@\spxentry{Installation}!VS\sphinxhyphen{}Code@\spxentry{VS\sphinxhyphen{}Code}}\ignorespaces 

\begin{savenotes}\sphinxattablestart
\centering
\phantomsection\label{\detokenize{software/vscodium:index-1}}\nobreak
\begin{tabulary}{\linewidth}[t]{|T|T|}
\hline
\sphinxstyletheadfamily 
\sphinxAtStartPar
Tastenkombination
&\sphinxstyletheadfamily 
\sphinxAtStartPar
Anmerkung
\\
\hline
\sphinxAtStartPar
\sphinxkeyboard{\sphinxupquote{Ctrl}}
&
\sphinxAtStartPar
Globale Konfig suchen
\\
\hline
\sphinxAtStartPar
\sphinxkeyboard{\sphinxupquote{Ctrl+f}}
&
\sphinxAtStartPar
Suchen
\\
\hline
\sphinxAtStartPar
\sphinxkeyboard{\sphinxupquote{Ctrl\sphinxhyphen{}h}}
&
\sphinxAtStartPar
Suchen und ersetzen
\\
\hline
\sphinxAtStartPar
\sphinxkeyboard{\sphinxupquote{Ctrl\sphinxhyphen{}Shift+f}}
&
\sphinxAtStartPar
In Dateien suchen
\\
\hline
\end{tabulary}
\par
\sphinxattableend\end{savenotes}


\subsection{Pro und Contra}
\label{\detokenize{software/vscodium:pro-und-contra}}

\begin{savenotes}\sphinxattablestart
\centering
\begin{tabulary}{\linewidth}[t]{|T|T|}
\hline
\sphinxstyletheadfamily 
\sphinxAtStartPar
Pro
&\sphinxstyletheadfamily 
\sphinxAtStartPar
Contra
\\
\hline
\sphinxAtStartPar
Übersichtlichkeit
&
\sphinxAtStartPar
läuft nicht ohne GUI
\\
\hline
\sphinxAtStartPar
minimaler Installationsaufwand
&
\sphinxAtStartPar
erhötes sicherheits risiko
\\
\hline
\sphinxAtStartPar
offline nutzbar
&\\
\hline
\end{tabulary}
\par
\sphinxattableend\end{savenotes}

\sphinxstepscope


\section{Zeal}
\label{\detokenize{software/zeal:zeal}}\label{\detokenize{software/zeal::doc}}
\index{Zeal@\spxentry{Zeal}}\ignorespaces 
\sphinxAtStartPar
Website: www.zealdocs.org


\begin{savenotes}\sphinxattablestart
\centering
\begin{tabulary}{\linewidth}[t]{|T|T|}
\hline
\sphinxstyletheadfamily 
\sphinxAtStartPar
Pro
&\sphinxstyletheadfamily 
\sphinxAtStartPar
Contra
\\
\hline
\sphinxAtStartPar
Dokumentation für 3. Person
&
\sphinxAtStartPar
Erstellung eigener Docsets ist aufwendig
\\
\hline
\sphinxAtStartPar
minimaler Installationsaufwand
&
\sphinxAtStartPar
ev. keine Inhalte zum Theme
\\
\hline
\sphinxAtStartPar
offline nutzbar
&\\
\hline
\end{tabulary}
\par
\sphinxattableend\end{savenotes}


\section{Installation}
\label{\detokenize{software/zeal:installation}}
\index{Installation@\spxentry{Installation}}\ignorespaces 
\def\sphinxLiteralBlockLabel{\label{\detokenize{software/zeal:index-1}}}
\begin{sphinxVerbatim}[commandchars=\\\{\}]
\PYG{c+c1}{\PYGZsh{} git (siehe doku zu git)}
\PYG{c+c1}{\PYGZsh{} emacs (siehe doku zu emacs)}
\PYG{c+c1}{\PYGZsh{} Python (siehe doku zu Python)}
\PYG{n}{pip} \PYG{n}{install} \PYG{n}{sphinx}
\PYG{n}{pip} \PYG{n}{install} \PYG{n}{sphinx}\PYG{o}{\PYGZhy{}}\PYG{n}{autobuild}
\PYG{n}{pip} \PYG{n}{install} \PYG{n}{furo} \PYG{c+c1}{\PYGZsh{} Theme}

\PYG{c+c1}{\PYGZsh{} für doc2dash}
\PYG{n}{pip} \PYG{n}{install} \PYG{n}{pipx}
\PYG{n}{pipx} \PYG{n}{install} \PYG{n}{doc2dash}
\end{sphinxVerbatim}


\section{Übersetzung}
\label{\detokenize{software/zeal:ubersetzung}}
\begin{DUlineblock}{0em}
\item[] Von HTML (Sphinx) zu Docset (Zeal/Dash)
\item[] Nutze C:/home/dokus/leon/env/Scripts/doc2dash
\end{DUlineblock}

\sphinxstepscope


\chapter{Latex to PDF}
\label{\detokenize{latexpdf/index:latex-to-pdf}}\label{\detokenize{latexpdf/index::doc}}
\sphinxstepscope


\section{Latex to PDF}
\label{\detokenize{latexpdf/latextopdf:latex-to-pdf}}\label{\detokenize{latexpdf/latextopdf::doc}}

\subsection{Aus der Sphinx Doku eine PDF erstellen}
\label{\detokenize{latexpdf/latextopdf:aus-der-sphinx-doku-eine-pdf-erstellen}}
\begin{sphinxVerbatim}[commandchars=\\\{\}]
\PYG{n}{make} \PYG{n}{latex}
\end{sphinxVerbatim}


\subsection{Installation}
\label{\detokenize{latexpdf/latextopdf:installation}}
\begin{DUlineblock}{0em}
\item[] Auf Windows
\item[] \sphinxhyphen{}PERL installieren
\item[]
\begin{DUlineblock}{\DUlineblockindent}
\item[] \sphinxurl{https://strawberryperl.com/}
\end{DUlineblock}
\item[] \sphinxhyphen{}MikTeX Package Manager installieren
\item[]
\begin{DUlineblock}{\DUlineblockindent}
\item[]
\begin{DUlineblock}{\DUlineblockindent}
\item[] \sphinxurl{https://miktex.org/download}
\end{DUlineblock}
\item[] \sphinxhyphen{}in der GUI unter ‚Pakete‘ latexmk installieren
\item[]
\begin{DUlineblock}{\DUlineblockindent}
\item[] \sphinxhyphen{}Im ZIT muss ein Proxy gesetzt werden \sphinxhyphen{}\textgreater{} Ändern und Verbindungseinstellungen
\end{DUlineblock}
\end{DUlineblock}
\end{DUlineblock}

\begin{sphinxVerbatim}[commandchars=\\\{\}]
\PYG{l+m+mf}{10.128}\PYG{l+m+mf}{.9}\PYG{l+m+mf}{.30}\PYG{p}{:}\PYG{l+m+mi}{80}
\end{sphinxVerbatim}


\subsection{latexmk ausführen}
\label{\detokenize{latexpdf/latextopdf:latexmk-ausfuhren}}
\sphinxAtStartPar
latexmk ist ein cmd programm und läuft in der Konsole

\sphinxAtStartPar
Um es im Latex file Ordner auszuführen öffnet man CMD der Befehl ist

\begin{sphinxVerbatim}[commandchars=\\\{\}]
\PYG{n}{latexmk}
\end{sphinxVerbatim}

\sphinxAtStartPar
Dann werden alle .tex Datein im Ordner berücksichtigt

\sphinxAtStartPar
Wenn man sicher gehen möchte einen .pdf Datei zu bekommen tippt man

\begin{sphinxVerbatim}[commandchars=\\\{\}]
\PYG{n}{latexmk} \PYG{o}{\PYGZhy{}}\PYG{n}{pdf}
\end{sphinxVerbatim}

\sphinxAtStartPar
Wenn man nur spezifische .tex Datein Konvertieren möchte

\begin{sphinxVerbatim}[commandchars=\\\{\}]
\PYG{n}{latexmk} \PYG{n}{datei}\PYG{o}{.}\PYG{n}{tex}
\end{sphinxVerbatim}

\sphinxAtStartPar
Wenn man dauerhaft seine Änderungen als PDF Konvertiert haben möchte

\begin{sphinxVerbatim}[commandchars=\\\{\}]
\PYG{n}{latexmk} \PYG{o}{\PYGZhy{}}\PYG{n}{pvc}
\end{sphinxVerbatim}

\sphinxAtStartPar
Um im Ordner aufzuräumen und alle temporären Datein zu löschen

\begin{sphinxVerbatim}[commandchars=\\\{\}]
\PYG{n}{latexmk} \PYG{o}{\PYGZhy{}}\PYG{n}{c}
\end{sphinxVerbatim}

\sphinxstepscope


\chapter{Was ich heute gerlernt habe}
\label{\detokenize{wihgh/index:was-ich-heute-gerlernt-habe}}\label{\detokenize{wihgh/index::doc}}
\sphinxstepscope


\section{September 2022}
\label{\detokenize{wihgh/2022-09:september-2022}}\label{\detokenize{wihgh/2022-09::doc}}
\sphinxstepscope


\chapter{Management}
\label{\detokenize{projektmanagement/index:management}}\label{\detokenize{projektmanagement/index::doc}}
\sphinxstepscope


\section{Projekt Management}
\label{\detokenize{projektmanagement/projektmanagement:projekt-management}}\label{\detokenize{projektmanagement/projektmanagement::doc}}
\sphinxstepscope


\chapter{Glossar}
\label{\detokenize{glossar:glossar}}\label{\detokenize{glossar::doc}}

\section{SBOM}
\label{\detokenize{glossar:sbom}}
\sphinxAtStartPar
Software Builds of Materials


\chapter{Anhang}
\label{\detokenize{index:anhang}}\begin{itemize}
\item {} 
\sphinxAtStartPar
\DUrole{xref,std,std-ref}{genindex}

\item {} 
\sphinxAtStartPar
\DUrole{xref,std,std-ref}{modindex}

\item {} 
\sphinxAtStartPar
\DUrole{xref,std,std-ref}{search}

\end{itemize}



\renewcommand{\indexname}{Stichwortverzeichnis}
\printindex
\end{document}